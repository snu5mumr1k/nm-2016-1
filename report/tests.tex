\newpage
\section{Тесты}
Проверка проводилась с помощью приложенного скрпита на python3.4, с использованием библиотеки numpy
\subsection{Небольшие тесты}
\subsubsection{Тест 1}
\begin{equation*}
    \begin{cases}
    2x_1 + 5x_2 + 4x_3 + x_4 = 20, \\
    x_1 + 3x_2 + 2x_3 + x_4 = 11, \\
    2x_1 + 10x_2 + 9x_3 + 7x_4 = 40, \\
    3x_1 + 8x_2 + 9x_3 + 2x_4 = 37.
    \end{cases}
\end{equation*}
Результат:
Рассчитано программой:
\begin{equation*}
    \begin{pmatrix}
        1 \\
        2 \\
        2 \\
        -0
    \end{pmatrix}
\end{equation*}
Рассчитано при помощи пакета numpy:
\begin{equation*}
    \begin{pmatrix}
        1.0 \\
        2.0 \\
        2.0 \\
        -3.7 \times 10^{-15}
    \end{pmatrix}
\end{equation*}

\subsubsection{Тест 2}
\begin{equation*}
    \begin{cases}
    6x_1 + 4x_2 + 5x_3 + 2x_4 = 1, \\
    3x_1 + 2x_2 + 4x_3 + x_4 = 3, \\
    3x_1 + 2x_2 - 2x_3 + x_4 = -7, \\
    9x_1 + 6x_2 + x_3 + 3x_4 = 2.
    \end{cases}
\end{equation*}
Результаты:
Матрица является вырожденной.

\subsubsection{Тест 3}
\begin{equation*}
    \begin{cases}
    2x_1 + x_2 + x_3 = 2, \\
    x_1 + 3x_2 + x_3 + x_4 = 5, \\
    x_1 + x_2 + 5x_3 = -7, \\
    2x_1 + 3x_2 - 3x_3 - 10x_4 = 14.
    \end{cases}
\end{equation*}
Результат:
Рассчитано программой:
\begin{equation*}
    \begin{pmatrix}
        1 \\
        2 \\
        -2 \\
        -0
    \end{pmatrix}
\end{equation*}
Рассчитано при помощи пакета numpy:
\begin{equation*}
    \begin{pmatrix}
        1.0 \\
        2.0 \\
        -2.0 \\
        -0.0
    \end{pmatrix}
\end{equation*}
\newpage
\subsection{Объёмные тесты}

Тестирование проводилось для матриц порядка $n = 100$, заданных формулой
\begin{equation*}
A_{ij} = 
    \begin{cases}
        q_M^{i + j} + 0.1 \cdot (j - i), i \ne j, \\
        \left(q_M - 1\right)^{i + j}, i = j.
    \end{cases}
\end{equation*}
где $q_M = 1.001 - 2 * M * 10^{-3}$, $ i, j = 1, \ldots n$.
$M$ было принято равным 6(по условию).
Элементы вектора свободных коэффициентов задавались формулой:
$$
    x \cdot \exp{\frac{x}{i}} \cdot \cos{\frac{x}{i}}
$$

Программа выполнялась для значений $X$ из промежутка с $0$ до $15$ с шагом $0.5$.
Для проверки решений использовался приложенный скрипт check.py, в котором считалась евклидова норма вектора невязки.
При увеличении $x$ это значение растёт как для обычного метода Гаусса, так и для его усовершенствованного варианта.
Причём выбор главного элемента даёт выигрыш до нескольких порядков (см. таблицу).
\newpage
\begin{tabular}{|l|c|c|r|}
    \hline
        $x$ & Без модификации & С модификацией & Разность \\ \hline
        0.0 &   0 & 0 & 0 \\ \hline
        0.5 &   6.8153e-11  & 4.70669e-14 & -6.81059e-11 \\ \hline
        1.0 &   2.48026e-10 & 1.48061e-13 & -2.47878e-10 \\ \hline
        1.5 &   3.1454e-10  & 1.16844e-13 & -3.14423e-10 \\ \hline
        2.0 &   2.92344e-10 & 2.49085e-13 & -2.92095e-10 \\ \hline
        2.5 &   1.79998e-09 & 1.05803e-12 & -1.79892e-09 \\ \hline
        3.0 &   5.69084e-09 & 3.3937e-12  & -5.68745e-09 \\ \hline
        3.5 &   1.05627e-08 & 6.3643e-12  & -1.05563e-08 \\ \hline
        4.0 &   1.60849e-08 & 4.44234e-12 & -1.60804e-08 \\ \hline
        4.5 &   1.0672e-08  & 2.01469e-12 & -1.067e-08 \\ \hline
        5.0 &   1.74529e-08 & 1.17426e-11 & -1.74412e-08 \\ \hline
        5.5 &   8.85728e-08 & 4.26294e-11 & -8.85301e-08 \\ \hline
        6.0 &   2.33569e-07 & 3.68016e-11 & -2.33532e-07 \\ \hline
        6.5 &   4.15609e-07 & 1.26384e-10 & -4.15483e-07 \\ \hline
        7.0 &   5.02028e-07 & 1.90374e-10 & -5.01838e-07 \\ \hline
        7.5 &   4.25878e-07 & 1.48411e-10 & -4.25729e-07 \\ \hline
        8.0 &   3.85135e-07 & 1.35395e-10 & -3.84999e-07 \\ \hline
        8.5 &   2.64374e-06 & 1.35466e-09 & -2.64239e-06 \\ \hline
        9.0 &   6.69058e-06 & 2.61472e-09 & -6.68797e-06 \\ \hline
        9.5 &   1.28617e-05 & 3.03868e-09 & -1.28587e-05 \\ \hline
        10.0 &  1.861e-05   & 2.94096e-09 & -1.86071e-05 \\ \hline
        10.5 &  1.85298e-05 & 1.42146e-08 & -1.85156e-05 \\ \hline
        11.0 &  4.33969e-07 & 2.79863e-10 & -4.3369e-07 \\ \hline
        11.5 &  5.94858e-05 & 5.69602e-08 & -5.94288e-05 \\ \hline
        12.0 &  0.000164127 & 7.69133e-08 & -0.00016405 \\ \hline
        12.5 &  0.000334385 & 1.47586e-07 & -0.000334238 \\ \hline
        13.0 &  0.000495664 & 3.49012e-07 & -0.000495315 \\ \hline
        13.5 &  0.000619153 & 2.65897e-07 & -0.000618887 \\ \hline
        14.0 &  0.000248584 & 8.04283e-08 & -0.000248504 \\ \hline
        14.5 &  0.000958917 & 6.01043e-07 & -0.000958316 \\ \hline
        15.0 &  0.00407568  & 1.69442e-06 & -0.00407399 \\ \hline
\end{tabular}
\label{tab:residual}
