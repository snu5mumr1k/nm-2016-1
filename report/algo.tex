\newpage
\section{Описание алгоритма решения}

Для решения поставленных задач будет использоваться метод Гаусса~--~
один из классических методов решения системы линейных алгебраических уравнений.
Метод подразумевает решение системы $Ax = f$ путём приведения $A$~--~матрицы системы
к ступенчатому виду (прямой ход) и выражения значений переменных (обратный ход).
\par

Итак, пусть $A$~--~матрица порядка $n \times n$, $f$~--~вектор-столбец свободных членов системы порядка $n$.
\par

Прямой ход состоит в послдовательном повторении трёх шагов для всех $i = 1 \ldots n$:
\begin{itemize}
    \item Из строчек $A$ с $i$ по $n$ выберем такую, что $A_{ii} \ne 0$ (в случае классичесского метода Гаусса) и
        такую, что $A_{ii} = \max\limits_{j=i \ldots n} A_{ji}$ (в случае метода Гаусса с выбором главного элемента по столбцу)
        и поменяем её местами с $i$-той. В случае, если после этого $A_{ii} = 0$, отметим, что матрица является вырожденной (значит, либо система несовместна, либо многообразие решений невырожденно)
        и перейдём к $i = i + 1$.
    \item Поделим теперь все элементы $i$-той строчки $A$ (а так же $i$-тую строчку правой части) на $A_{ii}$.
    \item Теперь из каждой $j$-той ($j = i + 1 \ldots n$) строчки $A$ и правой части вычтем $i$-тую, домноженную на $A_{ji}$.
        Теперь каждый элемент $i$-того столбца, стоящий ниже $i$-того равен нулю.
\end{itemize}
\par
После выполнения прямого хода матрица приведена к верхне-ступенчатому виду.
Причём если $r$~--~номер последней ненулевой строчки в матрице $A$, $r = rank(A)$.
Теперь если найдётся $f_j \ne 0$ ($j = r + 1 \ldots n$), то система несовместна.
В ином случае многообразие решений имеет размерность, равную $n - r$.
В частности, если $r = n$, решение единственно. В дальнейшем будем считать, что мы работаем именно с невырожденными матрицами,
то есть решение единственно и матрица имеет верхнетреугольный вид.
\par

Обратный ход включает повторение единственного шага для всех $i = r \ldots 1$:
\begin{itemize}
    \item Из каждой $j$-той строчки ($j < i$) вычитаем $i$-тую, домноженную на $A_{ji}$, не забыв проделать то же с правой частью.
\end{itemize}
\par

После завершения обратного хода невырожденная матрица $A$ станет единичной.
\par

Заметим, что от правой части уравнения требуется только поддерживать т.н. "элементарные преобразования":
\begin{enumerate}
    \item обмена двух рядов;
    \item умножения ряда на число;
    \item прибавления к одному ряду другого, домноженного на число.
\end {enumerate}
\par

В случае, если матрица невырожденная, а правая часть является вектором, как было описано выше, в результате работы алгоритма в правой части окажется искомый корень уравнения.
Если же в качестве правой части использовать единичную матрицу, то в результате работы алгоритма справа получим матрицу $A^{-1}$.
\par

Осталось научиться вычислять определитель матрицы. Для этого нужно отметить, что определитель меняется при проведении первых двух из перечисленных преобразований~--~
меняет знак при обмене двух рядов и, при умножении ряда на число $a$, делится на $a$.
