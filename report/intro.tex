\newpage
\section{Цель работы}

Изучить классический метод Гаусса решения системы линейных алгебраических уравнений.

\section{Постановка задачи}

Дана система уравнений $Ax = f$ порядка $n \times n$ с невырожденной матрицей $A$.
Написать программу, решающую систему линейных алгебраических уравнений заданного пользователем размера
($n$ -- параметр программы) методом Гаусса и методом Гаусса с выбором главного элемента.
\par

Предусмотреть возможность задания элементов матрицы системы и её правой части как во входном файле данных,
так и путём задания специальных формул.
\par

\section{Цели и задачи практической работы}
\begin{enumerate}
    \item Научиться решать заданную СЛАУ методом Гаусса и методом Гаусса с выбором главного элемента;
    \item Вычислить определитель матрицы $det(A)$;
    \item Вычислить обратную матрицу $A^{-1}$;
    \item Исследовать вопрос вычислительной устойчивости метода Гаусса (для матриц высоких порядков);
    \item Правильность решения СЛАУ подтвердить системой тестов.
\end{enumerate}
