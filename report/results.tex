\newpage
\section{Выводы}
Приведённые результаты тестов показывают, что уже при $n = 100$ нетрудно найти систему уравнений с $n$ переменными,
для которой простой метод Гаусса будет давать заметную (относительно исходных данных) ошибку.
Таким образом, метод вычислительно устойчивым не является.
\par

Приведённая модификация метода Гаусса~--~метод Гаусса с выбором главного элемента показывает лучшую на несколько порядков точность,
но видно, что если увеличить $n$, и этой точности станет недостаточно.
\par

Для решения этой проблемы можно выбирать максимальный элемент не из столбца, а из всей необработанной подматрицы.
Это позволит увеличить точность, но добавит асимптотической сложности и так не оптимальному алгоритму, поэтому подобное решение
будет зависеть уже от конкретных целей в соблюдении баланса "время выполнения - точность".
