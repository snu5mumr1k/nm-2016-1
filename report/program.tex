\newpage
\section{Описание программы}
Программа написана на C++11. \par
Краткое содержание реализации (подробнее -- см. код):
\begin{itemize}
    \item Чисто виртуальный класс BaseElementGenerator, предоставляющий интерфейс для задания системы уравнений;
    \item Класс ElementGenerator, реализующий заданные в условии формулы для получения очередного элемента матрицы и вектора;
    \item Класс EquationsSystem, описывающий систему уравнений. В качестве параметра конструктора принимает имя файла, содержащего систему или объект типа BaseElementGenerator.
        Имеет методы Solve(), Determinant() и Inverse(), осуществляющие решение системы, нахождение определителя матрицы коэффициентов и обращение матрицы коэффициентов.
        Поддерживает неявный параметр, определяющий, использовать или нет модификацию метода Гаусса.
    \item Чисто виртуальный класс BaseRightEquationsSystemPart, предоставляющий интерфес для задания правой части уравнения $Ax = f$ 
        (В частности с помощью него реализован подсчёт числа операций);
    \item Классы Matrix и Vector, представляющие из себя рализации квадратной матрицы и вектора соответственно. Оба класса отнаследованы от BaseRightEquationsSystemPart, а также 
        представляют стандартный набор операций для этих алгебраических объектов (перегружены операторы $*, *=, +, +=, -, -=$)
    \item Также реализованы вспомогательные функции Equal(double, double) и Zero(double), проверяющие числа на равенства и на равенство нулю, соответственно 
    и класс OperationsCounter, который позволяет посчитать количество операций при выполнении наивно реализованного алгоритма.
\end{itemize}
