\newpage
\section{Тесты}
Проверка проводилась с помощью приложенного скрпита на python3.4, с использованием библиотеки numpy
Установленная точность вычислений $0.001$
\subsection{Небольшие тесты}
\paragraph{Тест 1}
\begin{equation*}
    \begin{cases}
    2x_1 + 5x_2 + 4x_3 + x_4 = 20, \\
    x_1 + 3x_2 + 2x_3 + x_4 = 11, \\
    2x_1 + 10x_2 + 9x_3 + 7x_4 = 40, \\
    3x_1 + 8x_2 + 9x_3 + 2x_4 = 37.
    \end{cases}
\end{equation*}
Результат:
Рассчитано программой:
\begin{equation*}
    \begin{pmatrix}
        1.22884 \\
        1.88988 \\
        2.00155 \\
        0.0900723
    \end{pmatrix}
\end{equation*}
Рассчитано при помощи пакета numpy:
\begin{equation*}
    \begin{pmatrix}
        1.0 \\
        2.0 \\
        2.0 \\
        -3.7 \times 10^{-15}
    \end{pmatrix}
\end{equation*}

\paragraph{Тест 2}
\begin{equation*}
    \begin{cases}
    6x_1 + 4x_2 + 5x_3 + 2x_4 = 1, \\
    3x_1 + 2x_2 + 4x_3 + x_4 = 3, \\
    3x_1 + 2x_2 - 2x_3 + x_4 = -7, \\
    9x_1 + 6x_2 + x_3 + 3x_4 = 2.
    \end{cases}
\end{equation*}
Результаты:
Матрица является вырожденной.

\paragraph{Тест 3}
\begin{equation*}
    \begin{cases}
    2x_1 + x_2 + x_3 = 2, \\
    x_1 + 3x_2 + x_3 + x_4 = 5, \\
    x_1 + x_2 + 5x_3 = -7, \\
    2x_1 + 3x_2 - 3x_3 - 10x_4 = 14.
    \end{cases}
\end{equation*}
Результат:
Рассчитано программой:
\begin{equation*}
    \begin{pmatrix}
        0.999765 \\
        2.00009 \\
        -1.99996 \\
        -3.33812 \times 10^{-05}
    \end{pmatrix}
\end{equation*}
Рассчитано при помощи пакета numpy:
\begin{equation*}
    \begin{pmatrix}
        1.0 \\
        2.0 \\
        -2.0 \\
        -0.0
    \end{pmatrix}
\end{equation*}

\newpage
\subsection{Объёмные тесты}

Тестирование проводилось для матриц порядка $n = 100$, заданных формулой
\begin{equation*}
A_{ij} = 
    \begin{cases}
        q_M^{i + j} + 0.1 \cdot (j - i), i \ne j, \\
        \left(q_M - 1\right)^{i + j}, i = j.
    \end{cases}
\end{equation*}
где $q_M = 1.001 - 2 * M * 10^{-3}$, $ i, j = 1, \ldots n$.
$M$ было принято равным 6(по условию).
Элементы вектора свободных коэффициентов задавались формулой:
$$
    x \cdot \exp{\frac{x}{i}} \cdot \cos{\frac{x}{i}}
$$

Программа выполнялась для значений $x$ из промежутка с $2$ до $5$ с шагом $0.5$.
Значение $\omega$ подбиралось экспериментально для быстрейшей сходимости.
Для проверки решений использовался приложенный скрипт check.py, в котором считалась евклидова норма вектора невязки.
Погрешность проверки $0.001$.
При данных настройках метод верхней релаксации даёт большую погрешность, чем метод Гаусса, поскольку для ограничения 
времени выполнения была искусственно уменьшена точность вычислений.

