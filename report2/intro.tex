\newpage
\section{Цель работы}

Изучить классические итерационные методы (Зейделя и верхней релаксации), используемые для численного решения систем линейных алгебраических уравнений.
Изучить скорость сходимости этих методов в зависимости от выбора итерационного параметра.
\par

\section{Постановка задачи}
Дана система уравнений $Ax = f$ порядка $n \times n$ с невырожденной матрицей $A$.
Написать программу, решающую систему линейных алгебраических уравнений заданного пользователем размера
($n$ -- параметр программы), использующую численный алгоритм итерационного метода Зейделя:
\par
\begin{equation*}
    \left(D + A^{(-)}\right)\left(x^{k + 1} - x^k\right) + Ax^k = f
\end{equation*}
где $D, A^{(-)}$ - соответственно диагональная и нижняя треугольные матрицы, $k$ - номер текущей итерации.
\par
в случае использования итерационного метода верхней релаксации итерационный процесс имеет вид:
$$
    \left(D + \omega A^{(-)}\right)\frac{x^{k + 1} - x^k}{\omega} + Ax^k = f
$$
где $\omega$ - итерационный параметр (при $\omega = 1$ метод верхней релаксации переходит в метод Зейделя).

\section{Цели и задачи практической работы}
\begin{enumerate}
    \item Научиться решать заданную СЛАУ итерационным методом Зейделя;
    \item Разработать критерий остановки итерационного процесса, гарантирующий получение приближённого решения исходной СЛАУ с заданной точностью;
    \item Изучить скорость сходимости итераций к точному решению задачи;
    \item Правильность решения СЛАУ подтвердить системой тестов.
\end{enumerate}
