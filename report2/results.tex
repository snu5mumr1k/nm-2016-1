\newpage
\section{Выводы}
В результате выполненной работы установлено, что метод верхней релаксации позволяет решать системы линейных уравнений.
Но в классическом варианте для больших плохо обусловленных матриц от выбора параметра релаксации будет существенно зависеть сходимость.
\par

Плюсом метода является то, что он хорошо векторизуется и параллелизуется. В условиях однопоточного выполнения единственным плюсом данного
метода, как итерационного, является то, что он позволяет использовать минимум дополнительной памяти.
По времени же метод ограничен тем, что на каждом шаге алгоритма приходится заново вычислять невязку, а эта операция довольно затратна.
Использование других индикаторов сходимости (Например разность текущей гипотезы с предыдущей) приводит, на примере матрицы, заданной в 6.2,
к скорейшей сходимости, но гораздо меньшей точности результата.
\par

С другой стороны, данный метод довольно универсален и может быть применён в широком круге различных итерационных методов.
