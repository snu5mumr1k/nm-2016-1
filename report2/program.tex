\newpage
\section{Описание программы}
Программа написана на C++11. \par
Краткое содержание реализации, существенное в этой части работы (подробнее -- см. код):
\begin{itemize}
    \item Чисто виртуальный класс BaseElementGenerator, предоставляющий интерфейс для задания системы уравнений;
    \item Класс ElementGenerator, реализующий заданные в условии формулы для получения очередного элемента матрицы и вектора;
    \item Класс EquationsSystem, описывающий систему уравнений. В качестве параметра конструктора принимает имя файла, содержащего систему или объект типа BaseElementGenerator.
        Имеет метод SuccessiveOverRelaxation() с необязательным параметром $\omega$ ($1$ по умолчанию), который производит решение системы методом верхней релаксациии.
        Так же реализован метод RelaxationResidual(), подсчитывающий норму вектора невязки системы при её решении этим методом.
    \item Классы Matrix и Vector, представляющие из себя рализации квадратной матрицы и вектора соответственно. Оба класса отнаследованы от BaseRightEquationsSystemPart, а также 
        представляют стандартный набор операций для этих алгебраических объектов (перегружены операторы $*, *=, +, +=, -, -=$)
    \item Также реализованы вспомогательные функции Equal(double, double) и Zero(double), проверяющие числа на равенства и на равенство нулю, соответственно.
    \item Можно решить систему, заданную матрицей m, непосредственно обратившись к её методу SuccessiveOverRelaxation(), принимающему в качестве параметра Vector свободных коэффициентов
    и $\omega$
\end{itemize}
